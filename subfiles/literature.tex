\documentclass[../thesis.tex]{subfiles}

\begin{document}

\chapter{Tinjauan Pustaka}\label{chap:pstk}
\section{Kualitas Lingkungan Waterfront}
%ENVIRONMENT QUAALITY TOWARDS QUALITY OF LIFE
Saat ini kenyamanan yang dirasakan di waterfront menjadi tantangan besar terhadap kualitas hidup seseorang \citep{li2020}. Bahkan kualitas sebuah urban waterfront menjadi tolak ukur sebuah keberhasilan kota. Menurut \cite{lansing1969evaluation} kualitas dari sebuah lingkungan menyampaikan rasa kesejahteraan dan kepuasan kepada penduduk melalui \textit{karakteristik fisik, sosial maupun simbolis}. Dalam cakupan yang besar, \cite{smith1997quality} menjabarkan tabel prinsip dari kualitas dan kebutuhan yang \textit{urban enviornment } harus penuhi yaitu \textit{}liveability, karakter, penghubung, \textit{mobility}, kebebesan diri, dan keberagaman.

Beberapa tahun terakhir ini, kualitas \textit{urban environtment} menjadi perbincangan yang hangat dalam penelitian perkotaan. Mereka berpendapat dalam \cite{hubbard1996design} bahwa elemen kualitas yang sulit dipahami sangat penting dalam hubungan emosional yang kuat antara manusia dan linkungan binaan, yang mana dimediasi oleh rasa dan persepsi seorang (tentunya, ini beda terhadap setiap individu maupun grup dengan kebudayaan, nilai, dan latar belakang yang berbeda). Hingga saat ini, kualtias waterfront menjadi syarat pengembangan ekonomi kota; meningkatkan prospek pengembangan kota. Padahal dahulu ekonomi kota menjadi pendorong untuk kualitas waterfront
Perubahan kenyataan ini menjadi alasan yang kuat untuk mendorong kualitas fisik, sosial, estetika dan ekonomi suatu tepi laut.

%BAGIAN-BAGIAN KUALITAS LINGKUNGAN WATERFRONT
%part 1
Tujuan estetika tepi laut yang baik yang harus dicapai adalah seperti akses fisik, akses visual, pelestarian sejarah, dan rasa tempat dan kontinuitas\citep{richarda.lehmann1966}. Dengan bagian-bagian tersebut, warga kota akan melihat tepi laut lebih baik.
%part 2
Menurut \cite{tunbridge1992} menguraikan faktor kunci utama kesuksessan skema pengembangan waterfront adalah mixed-uses dan aktivitas untuk bersantai. Mixed-uses pada waterfront menjadi kata kunci alasan sebuah tepi laut sebagian besar hanya menarik orang-orang tertentu. Seperti perkataan \cite{gospodini2001} bahwa penggunaan kembali ruang berdimensi tunggal telah membatasi potensi pengembangan dan mencegah tempat itu untuk berintegerasi dengan pusat kota dan ruang terbuka yang berdekatan dengan tempat itu. Tepi laut yang menyenangkan menempatkan karakteristik multi dimensi, agar orang-orang menjadikannya sebagai tempat untuk menyeimbangkan kerja, rekreasi dan hidup. Keberagaman dimensi waterfront menambah aktivitas-aktivitas yang mungkin dilakukan.

%part 3
Kualitas yang ada pada lingkungan binaan waterfront memunculkan preferensi pengguna terhadap lingkungan tersebut. Preferensi penggunaan waterfront dapat diuraikan sebagai berikut:
\begin{inparaenum}
\item  Aktivitas bersantai, olahraga, dan laut bertujuan untuk mengembangkan tepi pantai ke area rekreasi \citep{breen1994waterfronts}. \cite{gospodini2009} menyebutnya sebagai \textit{`Popular leisure epicentres'}.
\item Jalur pejalan kaki, akuarium, ekologi, dan lahan parkir untuk mengubah tepi laut sebagai area lingkungan(\textit{environment areas}) \citep{costa1990logical}.
\item Aktivitas perusahaan, bisnis, rumah sakit, dan bank yang mengubah tepi pantai mnejadi lokasi finansial \citep{hoyle1999scale} \citep{hoyle2000confrontation}. \cite{gospodini2009} sering menyebutnya \textit{`entrepreneurial epicentres'}
\item Rumah mewah, bertujuan untuk menjadikan tepi pantai sebagai area perumahan \citep{dong2004waterfront}.
\item Bangunan-bangunan pelestarian sejarah meliputi hotel, restaurant, teater bahkan sungai untuk menjadikan tepi pantai sebagai kawasan \textit{heritage} \citep{macleod1999}. Atau \cite{gospodini2009} menyebutnya \textit{`high-culture epicentres'}.
	\end{inparaenum}

%part 4
	Dalam pembangunan berskala besar terhadap redevolpment waterfront untuk mengundang acara internasional. Kota Toronto berinisiatif untuk mengembangkan 6 pengembangan besar, berikut ini: \begin{inparaenum}\item Membangun tepi laut untuk kenyamanan publik \item akomodasi bisnis, pegawai dan ekonomi baru, \item Mengembangkan jaringan transportasi yang komprehensive, \item menyediakan lingkungan yang bersih \item Mengatur ulang dan integerasi untuk koridor Expresswa, dan \item Membuat tepi laut untuk Acara Olympic Games 2008\end{inparaenum}\citep{white2016}. Berbeda dengan \citep{mostafa2017} yang meringkas dampak urban dan sosial dari tepi laut yang mengungkapkan kebutuhan didominasi oleh : 1. pelayanan 2. taman 3. aktivitas 4. Shading 5. parkir 6. kafe dan rekreasi.
Dari hasil studi pustaka tentang kualitas lingkungan, penulis mendapatkan bahwa kualitas waterfront dapat menghasilkan aktivitas yang sangat beragam. Sehingga dari \textit{milestone} Jl. Pinggir Laut (baca: waterfront) ini mempertanyakan aktivitas apa saja yang mungkin terjadi dan hubungan dari kedua variabel.

\section{Fitur Binaan }

Salah satu aspek yang mendorong masyarakat untuk pergi ke sebuah \textit{public space} adalah pertama karena dekat, kedua karena fasilitas yang tersedia\textit{(amenities)}. Hasil penelitian dari \cite{campbell2016social} menunjukkan sekitar 24\% responden mengatakan bahwa mereka mengunjungi sebuah taman karena fasilitasnya. \textit{Amenities} termasuk infrasturktur ruang publik, seperti alat bermain, bangungan, fasilitas rekreasi, pusat alam\textit{(nature center)}. \textit{Aminities} atau \textit{built feature} adalah elemen penting dalam sebuah kawasan publik dan mendorong keaktifan didalam sebuah lingkungan.
Lingkungan yang memberikan ruang terhadap aktivitas fisik dapat meningkatkan kohesi sosial dan kemakmuran ekonomi dari sebuah lingkungan dalam hal penyimpanan energi, \textit{expenditure reduction} dan peningkatan kesehatan\citep{dovey2020walkability,klann2019translating}.

%many feture for support an activity and more
%walking activity
Salah satu aktivitas\textit{(outdoor activity)} yang menjadi perhatian peneliti belakangan ini adalah kecenderungan berjalan kaki diruang publik.
Apabila sebuah kota memperhatikan sebuah lingkungan binaan dengan baik seperti merancang fitur area yang padat dan mixed-used, lapangan ruang terbuka hijau, mendedikasikan infrstruktur untuk pejalan kaki maka akan meningkatkan kecendurangan untuk berjalan kaki\citep{cheng2020examining,cao2010neighborhood,cerin2014ageing}. Menurut \citep{cerin2014ageing} beberapa elemen\textit{(feature)} lingkungan yang berasosiasi terhadap aktivitas berjalan kaki adalah level dari urbanisasi, infrastruktur pedestrian, estetika lingkungan dan akses terhadap layanan tujuan(transportasi publik, ruang terbuka, dan toko-toko). Sementara \cite{ramakreshnan2020motivations} menyebut konektivitas dan aksesibilitas jalan merupakan faktor utama kecenderungan berjalan. Berbeda dengan \cite{liu2020impact} menyimpulkan bahwa pedestrian memiliki kecenderungan yang berbeda terhadap lingkungan binaan.
Hal ini menunjukkan bahwa satu aktivitas dapat didorong oleh berbagai macam elemen-elemen desain dari sebuah tempat dan berbeda dari setiap individu.

%aksesibilat pekerjaan
Selain dari berjalan, fitur binaan juga dapat mempengaruhi aksesibilitas pekerjaan di pusat kota yang padat ketimbang di tempat lain seperti pedesaan dan kota yang jarang penduduk\citep{zhu2020built}.
%transport actity
Dalam satu penelitan terkait built environment. Penelitian \citep{yu2019exploring}  menyebutkan bahwa karakteristik lingkungan binaan dihitung di kawasan unit lingkungan, termasuk kepadatan penghuni, jarak untuk transit, dan pemberhetian bis sepanjang 500m, berpengaruh signifikan terhadap sikap berpergian \textit{travel behaviour} termasuk jarak berpergian, waktu tempuh berpergian dan pilihan mode transit. Hubungan ini menjadi indikasi kekuatan fitur binaan dalam kehidupan masyarakat sebuah perkotaan, khususnya pada aktivitas mereka.


\section{Outdoor Activity}

Untuk mengetahui dampak dari suatu lingkungan terhadap kesehatan \cite{lachowycz2013towards} menggunakan apa yang disebut dengan mediasi; mekanisme apa yang melatarbelakangi hal tersebut. Hal yang menariknya adalah pengaruh kesehatan berdasarkan pendekatan lingkungan menggambarkan bahwa perilaku orang bergantung pada sebuah desain dan konteks dari sekitarnya \citep{cohen2010parks}. Melalui pendekatan lingkungan, karakteristik dari sebuah tempat dan interaksi individu terhadap karakteristik tersebut dapat menjadi pendorong untuk individu terlibat dalam sebuah aktivitas. Meskipun demikian, ilmu untuk menghubungkan \textit{setting} sebuah tempat dan aktivitas fisik masih kurang jelas kondisi dan karakterisitik\textit{(built feauture)} apa yang sebenarnya menarik untuk mendorong penggunaan ruang publik \citep{rull2005prescription,cohen2010parks}.

Penggunaan taman\textit{(park usage)} diinterpetasikan kemampuan sebuah taman mengakomodasi aktivitas didalamnya. Menurut \cite{cohen2010parks} faktor kunci yang dapat meningkatkan aktivitas didalamnya adalah karakteristik fisik dan sosial. Sebaliknya faktor yang dapat menghambat penggunaan sebuah taman adalah presepsi keamanan \citep{molnar2004unsafe,gomez2004violent,centers1999neighborhood}. Penyelenggaran aktivitas dalam sebuah taman juga dipengaruhi oleh intervensi lain seperti ketakutan \cite{roman2013pathways}.
Hal ini didukung dengan adanya teori \textit{`broken windows'}\citep{kelling1997fixing} yang menjadi perdebatan oleh ahli kriminal dan ahli sososiologi bahwa keruskan\textit{(disorder)} dapat menimbulkan aktivitas kriminal dan mengurangi aktivtias berjalan\textit{(walking)}. Akan tetapi penelitian \citep{cohen2010parks} menggubris teori tersebut, bahwa tidak ada hubungan presepsi keamanan terhadap penggunaan taman. Taman yang 100\% aman tidak serta merta memfasilitasi penggunaannya.

Aktivitas itu sendiri dapat menjadi sangat beragam, seperti contohnya aktivitas pada orangtua\textit{(elderly)} kebanyakan mungkin akan bergantung pada aktivitas pasif ketimbang aktivitas aktif. Sementara aktivitas orang dewasa\textit{(young adult)} akan cenderung ke aktivitas fisik\textit{(active)}. Aktivitas luar juga dapat berarti hubungan individu terhadap lainnya. Sebagai cerminan kebutuhan manusia untuk merasakan berhubungan dengan lainnya dan bagian dari suatu kelompok\citep{junot2017passion}. Lebih lanjut, aktivitas juga didasari oleh kebutuhan manusia menjadi bagian dari alam\citep{junot2017passion,nabhan1993loss}. Bersama dengan alam\textit{(nature)} dipertimbangkan sebagai faktor utama dalam perilaku lingkungan\textit{(environmental behaviours)}\citep{capaldi2014relationship,dutcher2007connectivity,kals1999emotional,mayer2004connectedness}




\section{Kerangka Penelitian}


Dari hasil tinjauan pustaka peneliti menyusun kerangka penelitian berdasarkan variabel-variable yang layak diteliti.


\begin{figure}[htbp]
\centering
\begin{tikzpicture}[node distance=2cm]

	\node (tit) [startstop, text width= 5cm] {Fitur Fisik Binaan pada Aktivitas Luar Jl. Pinggir Laut};

	\node (va1) [startstop, below of=tit, text width=5cm, xshift=-3cm] {Variabel Bebas\\ Fitur Fisik Binaan};

	\node (va2) [startstop, below of=tit, text width=5cm, xshift=3cm] {Variabel Tergantung\\ Aktivitas Luar};

	\node (de1) [startstop, below of=va1, text width=5cm, yshift=-2cm] {
		\textbf{Sub Variabel Bebas}\\
		- Elemen Jalan \\
		- Kualitas Jalan \\
		- Elemen Tempat Duduk \\
		- Kualitas Tempat Duduk \\
		- Elemen Alami \\
		- Kualitas Alami \\
		- Fasilitas \& Aminities \\
		- Estetika \\

	};
	\node (de2) [startstop, below of=va2, text width=5cm, yshift=-2cm] {
			\textbf{Sub Variable Tergantung}\\
		- Aktivitas relaxsasi\\
		- Aktivitas fisik\\
		- Travel aktif\\
		- Interaction with wildlife and nature\\
		- Interaksi sosial\\
		- Partisipasi di aktivitas grup\\
		};
\draw [arrow] (tit) -| (va1);
\draw [arrow] (va1) -- (de1);
\draw [arrow] (tit) -| (va2);
\draw [arrow] (va2) -- (de2);

\end{tikzpicture}
\caption{Alur Pikir}
\end{figure}

%\onlyinsubfile{\biblio}
\end{document}
